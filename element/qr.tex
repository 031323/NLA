\documentclass[12pt]{article}
\usepackage[hscale=0.8,vscale=0.8]{geometry}
\usepackage{amsmath}
\usepackage{amssymb}

\setlength{\parskip}{1em}

\pagenumbering{gobble}

\title{Quarterly Work Report}
\date{April-July 2020}
\author{Utkarsh Rajput \\\\{Supervisor: Prof. Sarvesh Kumar}}
\begin{document}
	\maketitle
	I continued my coursework for this semester in this period.
	
	For the Finite Element Analysis, I studied the mixed finite element method. It included the abstract problem and existence and uniquness of its solution. Next, the Lax Miligram theorem which is required for the above. Then, the discrete mixed formulation, the Inf-Sup condition for uniqueness and existence of the solution for this problem. Then, equivalence of the abstract problem to the saddle point problem was discussed. The extension of the problem for nearly incompressible materials was discussed. Later, the error estimates for the mixed finite element methods and perturbations of the problem, and finally applications to Poisson's equation and Navier Stokes equation.
	
	I also studied the necessity of the nonconforming finite element method for certain problems, the second Strang's Lemma which gives an error estimate for this method.
	
	For the Advanced PDE, I studied Sobolev Spaces, the Friedrichs theorem which gives approximation by smooth functions, the extension operator from $\Omega$ to $\mathbb{R}^n$, Stampacchia theorem, extension theorems which give the existence of extension operators for various cases, Poincare's inequality, imbedding theorems giving the imbedding properties of the Sobolev spaces, mainly the Sobolev inequality.
	
	For the Advanced Function Analysis course, I studied weak convergence in normed spaces, reflexive spaces, bounded operators on Banach spaces, bounded inverse theorem, self adjoint and symmetric operators, compact operator, spectral theory, properties of normal, unitary and self-adjoint operators, positive operators.

\end{document}
