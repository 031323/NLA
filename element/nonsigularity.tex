\documentclass[12pt]{article}
\usepackage[hscale=0.8,vscale=0.8]{geometry}
\usepackage{amsmath}
\usepackage{amssymb}
\usepackage{listings}
\usepackage{tikz}
\usepackage{color}

\definecolor{dkgreen}{rgb}{0,0.6,0}
\definecolor{gray}{rgb}{0.5,0.5,0.5}
\definecolor{mauve}{rgb}{0.58,0,0.82}

\lstset{frame=tb,
  aboveskip=3mm,
  belowskip=3mm,
  showstringspaces=false,
  columns=flexible,
  basicstyle={\small\ttfamily},
  numbers=none,
  numberstyle=\tiny\color{gray},
  keywordstyle=\color{blue},
  commentstyle=\color{dkgreen},
  stringstyle=\color{mauve},
  breaklines=true,
  breakatwhitespace=true,
  tabsize=4
}

\setlength{\parskip}{1em}

\pagenumbering{gobble}

\begin{document}
	Let $\{\phi_i\}_{i=1,2,...,n}$ be a basis of $V_h$. Let $a(.,.)$ be an inner product on $V_h$. Then, the matrix $\textbf{B}=[a(\phi_i,\phi_j)]_{n\times n}$ is non-singular.
	
	 \textbf{Proof:}
	 
	Assume that $\{c_1,c_2,...,c_n\}\subset \mathbb{R}$ which satisfies the following.
	$$\textbf{B}
	 \begin{bmatrix}
	c_1\\c_2\\\vdots\\c_n
	\end{bmatrix}=0$$
	
	$\implies \forall i \in \{1,2,...,n\},\;\;\;\;\;\;\;\; \sum\limits_{j=1}^n c_j a(\phi_i,\phi_j)=0 $
	
	$\implies \forall i \in \{1,2,...,n\},\;\;\;\;\;\;\;\; a(\phi_i,\sum\limits_{j=1}^n c_j\phi_j)=0 $
	
	As $\{\phi_i\}$ is a basis of $V_h$ and the above is true for all $i$, we get that $\sum_{j=1}^n c_j\phi_j=0$. But $\{\phi_j\}$ are linearly independent. Hence $c_j=0\;\;\;\forall j \in \{1,2,...,n\}$. Thus, the columns of $\textbf{B}$ are linearly independent and hence $\textbf{B}$ is non-singular.
	
\end{document}
