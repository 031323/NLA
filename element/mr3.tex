\documentclass[12pt]{article}
\usepackage[hscale=0.8,vscale=0.8]{geometry}
\usepackage{amsmath}
\usepackage{amssymb}

\setlength{\parskip}{1em}

\pagenumbering{gobble}

\title{Monthly Work Report}
\date{October 2020}
\author{Utkarsh Rajput \\\\{Supervisor: Prof. Sarvesh Kumar}}
\begin{document}
	\maketitle
	I implemented the convergence test for the Finite Element Method with problem with known solution. We compute the approximate error in the $L^{\infty}$ norm, that is, $||u-u_h||_{L^{\infty}}$ for given $h$. This can be approximated by $max_{n\in N} |u(n)-u_h(n)|$ where $N$ is the set of nodal points of the mesh.
	
	I studied and implemented the Eriksson-Johnson error indicator. Error indicators are used for adaptive mesh refinement. In adaptive mesh refinement, those parts of the mesh on which relative error is higher are refined more. This helps in faster convergence of the solution. For piecewise linear elements, error is higher at the points where the gradient of the solution is changing rapidly, that is, the second order derivatives are higher. The Eriksson-Johnson error indicator approximates the magnitude of the second order derivaties. Note that piecewise linear functions are not in $H^2$.
\end{document}
