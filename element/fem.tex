\documentclass{beamer}
\usepackage[english]{babel}
\usepackage{biblatex}
\usepackage{tikz}
\usepackage{color}

\def\R{\mathbb{R}}
\newcommand{\p}{ }

\newtheorem{gs}{General Solution}

\newcommand{\cb}{\color{blue}}
\newcommand{\cred}{\color{red}}
\newcommand{\cg}{\color{green}}
\newcommand{\cbr}{\color{Brown}}
\newcommand{\cm}{\color{Magenta}}
%\newcommand{\cb}{\color{blue}}
\newcommand{\dsp}{\displaystyle}

\newcommand{\fby}{\fcolorbox{blue}{yellow}}
\newcommand{\fbw}{\fcolorbox{blue}{white}}
\def \stackt{{\stackrel{.}{.\;.}\;\;}}
\def \stackb{{\stackrel{.\;.}{.}\;\;}}

\def\diam{\mathrm{diam}\,}
\def\p{\partial}
\def\O{\Omega}
\def\R{\mathbb{R}}
\def\cE{\mathcal{E}}
\def\cT{\mathcal{T}}
\def\cB{\mathcal{B}}
\def\cA{\mathcal{A}}
\def\cN{\mathcal{N}}
\def\Sides{\mathcal{E}_h}
\def\Jump#1{\Big[\!\!\Big[#1\Big]\!\!\Big]}
\def\mJump#1{\big[\hskip -3pt\big[#1\big]\hskip -3pt\big]}
\def\jump#1{[\![#1]\!]}
\def\Mean#1{\Big\{\!\!\!\Big\{#1\Big\}\!\!\!\Big\}}
\def\mMean#1{\big\{\hskip -4pt\big\{#1\big\}\hskip -4pt\big\}}
\def\mean#1{\{\!\!\{#1\}\!\!\}}
\def\cT{\mathcal{T}}
\def\cE{\mathcal{E}}
\def\ssT{{\scriptscriptstyle T}}
\def\sjump#1{[\hskip -1.5pt[#1]\hskip -1.5pt]}
\def\d{\displaystyle}
\newcommand\Blesssim{\;\raise 2pt\hbox{$<$}\hskip -16pt
  \lower 9pt\hbox{$\sim$}\;}
  \newcommand{\pa}{\partial}
  \newcommand{\del}{\bigtriangleup}
%\newcommand\lesssim{\;\raise 2pt\hbox{$<$}\hskip -14pt
%  \lower 8pt\hbox{$\sim$}\;}
\newcommand\alesssim{\raise 2pt\hbox{$<$}\hskip -14pt
  \lower 8pt\hbox{$\sim$}\;}
\def\deq{\,\raise 4pt\hbox{\tiny def}\hskip -10pt\lower
 4pt\hbox{$=$}\;}
\def\tbar{|\!|\!|}
\def\therefore{\hbox{\bf .}\hskip 2pt\hbox{\bf .}
 \hskip -12.5pt\raise 6pt\hbox{\bf .}\hskip 16pt}
\def\so{{\scriptscriptstyle 1}}
\def\st{{\scriptscriptstyle 2}}
\def\sth{{\scriptscriptstyle 3}}
\def\sz{{\scriptscriptstyle 0}}
\def\tR{{R'}}
\def\sf{{\scriptscriptstyle 4}}
\def\bE{\mathbb{E}}
\def\bA{\mathbb{A}}
%
\def\bu{\boldsymbol{u}}
\def\ba{\boldsymbol{a}}
\def\bsi{\boldsymbol{\sigma}}
\def\btau{\boldsymbol{\tau}}
\def\buz{\mathaccent'27{\bu}}
\def\bzz{\mathaccent'27{\bz}}
\def\bq{\boldsymbol{q}}
\def\bw{\boldsymbol{w}}
\def\bW{\boldsymbol{W}}
\def\bz{\boldsymbol{z}}
\def\bn{\boldsymbol{n}}
\def\bof{\boldsymbol{f}}
\def\curl{\nabla\times}
\def\div{\nabla\cdot}
\def\curlh{\nabla_h\times}
\def\ncross{{\bn\times}}
\def\HCurlz{H_0(\mathrm{curl};\O)}
\def\HDivz{H(\mathrm{div}^0;\O)}
\def\HDiv{H(\mathrm{div};\O)}
\def\HCD{\HCurlz\cap\HDivz}


\addbibresource{fem.bib}
\title{Finite Element Method for Elliptic Equations}
\subtitle{Presentation}
\author{Utkarsh Rajput\linebreak SC19D035}
\setbeamertemplate{bibliography item}{\insertbiblabel}
\setbeamertemplate{footline}{\hfill\usebeamertemplate***{navigation symbols}}
%\usetheme{lucid}
\begin{document}
	\frame{
		\titlepage
	}
	\frame{
\frametitle{Introduction}
The Finite Element Method (FEM) is a numerical techinique
\begin{itemize}
	\item for solving problems which are described by Ordinary Differential Equations (ODE) or Partial Differential Equations (PDE) with appropriate boundary initial conditions
	\item or to solve problems that can be formulated as a functional minimization
\end{itemize}

Fisrt suggested by Courant\footfullcite{courant} in 1943.
}
\frame{
\frametitle{Why FEM?}
\begin{itemize}
\item
 Greater flexibility to model complex geometries.\\ 
\item
 Can handle general boundary conditions.\\ 
\item
 Clear structure and versatility helps to construct general
purpose software for applications.\\ 
\item
 Has a solid theoretical foundation which gives added
reliability and makes it possible to mathematically analyze and estimate
the error in the approximate solution.\\
\end{itemize}
}
\frame{
		\frametitle{The method}
		
		The main steps involved are:
		
		1. Variational Formulation
		
		2. Discretization of the problem
		
		3. Converting to a system of linear equations
		
		4. Solving the system of linear equations
		
	}
	\frame{
		\frametitle{Poisson's Equation}
		$-\Delta u = f $ in $ \Omega$
		\linebreak\linebreak
		Dirichlet Condition: $u=0$ on $\Gamma \subset \partial \Omega$
		\linebreak\linebreak
		Neumann Condition: $\frac{\partial u}{\partial \nu}=0$ on $\partial \Omega \setminus \Gamma$
		\linebreak\linebreak
		Where $\frac{\partial u}{\partial \nu}=\nu \cdot \triangledown u $ and $\nu$ is the outward unit normal vector to $\partial \Omega$.
	}
	\frame{
		\frametitle{Variational Formulation}
		$V:=\{ v \in H^{1}(\Omega):v|_{\Gamma}=0\}$
		\linebreak\linebreak
		$a(u,v):=\int\limits_{\Omega}\triangledown u \cdot \triangledown v$ $ dx$
		\linebreak\linebreak
		$(f,v):=\int\limits_{\Omega} f v$ $ dx$
		\linebreak\linebreak
		Find $u \in V$ satisfiying $a(u,v)=(f,v)$  $ \forall v \in V$.
	}
	\frame{
		\frametitle{Galerkin Approximation Problem}
		Let $V_h \subset V$ be a finite dimensional subspace.
		\linebreak
		Find $u_h \in V_h$ such that $a(u_h,v)=(f,v)$ $\forall v \in V_h$.
		\linebreak\linebreak
		This is well posed because $a(.,.)$ is continuous elliptic bilinear form on $V_h$ and $(f,.)$ is continuous linear form. Thus Lax-Miligram theorem applies.
	}
	\frame{
		\frametitle{Galerkin Approximation Problem}
		Solve $[ a(e_j,e_i) ]_{n \times n} [ u_i ]_{n \times 1}=[(f,e_i)]_{n \times 1}$.
		\linebreak
		Where $n$ is the dimension of $V_h$ and $\{e_i\}$ is a basis of $V_h$.
		\linebreak\linebreak
		Then $\sum\limits_{i=1}^n u_i e_i$ is the solution of the Galerkin approximation problem.
	}
	\frame{
	{\cb Triangulation of $\bar{\Omega}$:
}
 

By an admissible triangulation $ \tau_h $ of a closed domain $\bar{\Omega} \subset \R^2$, we mean a subdivision of $\bar{\Omega}$ into closed triangles or rectangles or quadrilaterals denoted by $ \{ T_i \} $  such that
 
\begin{eqnarray}
& & {  (i) \; \bar{\Omega} = \displaystyle{\cup} T_i } \nonumber \\  
& &{  (ii) \; T_i \cap T_j =} \left \{ \begin{array}{l}
{  \phi} \\  
\mbox{  common vertex } \; \qquad \mbox{   for } {  i \neq j} \\  
\mbox{  common side}
\end{array} \right. \\  
& & {  (iii) \; T_i^{\circ} \cap T_j^{\circ} = \phi \; } \mbox{   for } {  i \neq j} \nonumber
\end{eqnarray}

	}

	\frame{
		\frametitle{Triangular Piecewise-polynomial Elements for 2-D Domain}
	For simplicity, we shall assume that $ \partial \Omega = \Gamma $ is a
polygonal curve, i.e., $\Omega$ is a polygonal domain. If $\Gamma$ is
curved, we may first approximate $\Gamma$ with a polygonal curve.
	}
	\frame{
		\frametitle{Triangular Piecewise-polynomial Elements for 2-D Domain}
		The domain $\Omega$ is approximated by a triangular mesh.
		\linebreak\linebreak
		On each triangle, the space of $k$ degree polynomials in two variables is $(k+1)(k+2)/2$ dimensional.
		\linebreak\linebreak
		In a Lagrange triangle, there are $3$ evaluation points on the vertices, $3(k-1)$ on the edge interiors and the remaining in the interior of the triangle.
	}
	
	\frame{
	\frametitle{An example}
		We take $V_h$ to be the space of continuous piecewise linear functions on $\bar{\Omega}$ which are linear on each triangle of the triangulation.
		\linebreak\linebreak
		Now we need to fix a basis for $V_h$. Here we use the nodal basis.
	}
	\frame{
	\frametitle{Nodal Basis}
  For $f \in V_h$, and $T\in \{T_i\}$, $f|_T$ would be of the form $$f|_T (x,y)=a_1 x+a_2 y+a_3$$
	Consider the triangle as follows.
	
	\begin{tikzpicture}
		\draw node[anchor=north east]{$(x_1,y_1)$} (0,0) --  (3,0) node[anchor=north west]{$(x_2,y_2)$} -- (0,3) node[anchor=south east]{$(x_3,y_3)$} -- (0,0);
		\draw[black,fill=black] (0,0) circle (.3ex);
		\draw[black,fill=black] (3,0) circle (.3ex);
		\draw[black,fill=black] (0,3) circle (.3ex);
	\end{tikzpicture}
	}
	\frame{
	\frametitle{Nodal Basis}
	
	Let $\{N_1,N_2,N_3\}$ be a subset of the space of linear funcionals on $P=V_h|_T$ defined as follows.
	\linebreak\linebreak
	$N_1(f)=f(x_1,y_1); \;\;\; N_2(f)=f(x_2,y_2); \;\;\; N_3(f)=f(x_3,y_3);$
	\linebreak\linebreak
	The nodal basis for $P$ is $\{\phi_1,\phi_2,\phi_3\}$ such that $N_i(\phi_j)=\delta_{ij} \;\;\; \forall i,j \in \{1,2,3\}$.
	}
	\frame{
	\frametitle{Nodal Basis}
	The nodal basis for $V_h$ is constructed as follows.
	\linebreak\linebreak
	For each node, the basis function is defined on each triangle $T$ as
	\begin{itemize}
	\item $0$ if the node does not lie in the closure of the $T$
	\item the nodal basis function of that node in $V_h|_T$ otherwise
	\end{itemize}
	
	Now, we can solve $[ a(e_j,e_i) ]_{n \times n} [ u_i ]_{n \times 1}=[(f,e_i)]_{n \times 1}$ to get the solution $\sum\limits_{i=1}^n u_i e_i$.
\nocite{*}
	}
\printbibliography
\end{document}
