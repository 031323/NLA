\documentclass{beamer}

\title{Finite Element Method}
\subtitle{Presentation}

%\usetheme{lucid}
\begin{document}
	\frame{
		\titlepage
	}
	\frame{
		\frametitle{Poisson's Equation}
		$-\Delta u = f $ in $ \Omega$
		\linebreak\linebreak
		Dirichlet Condition: $u=0$ on $\Gamma \subset \partial \Omega$
		\linebreak\linebreak
		Neumann Condition: $\frac{\partial u}{\partial \nu}=0$ on $\partial \Omega \setminus \Gamma$
		\linebreak\linebreak
		Where $\frac{\partial u}{\partial \nu}=\nu \cdot \triangledown u $ and $\nu$ is the outward unit normal vector to $\partial \Omega$.
	}
	\frame{
		\frametitle{Variational Formulation}
		$V:=\{ v \in H^{1}(\Omega):v|_{\Gamma}=0\}$
		\linebreak\linebreak
		$a(u,v):=\int\limits_{\Omega}\triangledown u \cdot \triangledown v$ $ dx$
		\linebreak\linebreak
		$(f,v):=\int\limits_{\Omega} f v$ $ dx$
		\linebreak\linebreak
		Find $u \in V$ satisfiying $a(u,v)=(f,v)$  $ \forall v \in V$.
	}
	\frame{
		\frametitle{Galerkin Approximation Problem}
		Let $V_h \subset V$ be a finite dimensional subspace.
		\linebreak
		Find $u_h \in V_h$ such that $a(u_h,v)=(f,v)$ $\forall v \in V_h$.
		\linebreak\linebreak
		Then, by Céa's Theorem, $$ ||u-u_h||\leq \frac{C}{\alpha} \min_{v \in V_h} ||u-v||_V$$
	}
	\frame{
		\frametitle{Galerkin Approximation Problem}
		Solve $[ a(e_j,e_i) ]_{n \times n} [ u_i ]_{n \times 1}=[(f,e_i)]_{n \times 1}$.
		\linebreak
		Where $n$ is the dimension of $V_h$ and $\{e_i\}$ is a basis of $V_h$.
		\linebreak\linebreak
		Then $\sum\limits_{i=1}^n u_i e_i$ is the solution of the Galerkin approximation problem.
	}
	\frame{
		\frametitle{Triangular Piecewise-polynomial Elements for 2-D Domain}
		The domain $\Omega$ is approximated by a triangular mesh.
		\linebreak\linebreak
		On each triangle, the space of $k$ degree polynomials in two variables is $(k+1)(k+2)/2$ dimensional.
		\linebreak\linebreak
		In a Lagrange triangle, there are $3$ evaluation points on the vertices, $3(k-1)$ on the edge interiors and the remaining in the interior of the triangle.
	}
	\frame{
		\frametitle{$P_1$ Nonconforming Element}
		In the $P_1$ nonconforming element, the functions are linear on each triangle and the three evaluation points are at the midpoints of the edges.
		\linebreak\linebreak
		Thus, the functions are in general discontinuous at the edges except at the midpoints.
	}
	\frame{
		\frametitle{$P_1$ Nonconforming Element}
		Let $V=H_{o}^{1}(\Omega)$.
		\linebreak\linebreak
		Then in the $P_1$ nonconforming element, $V_h$ is not a subspace of $V$.
		\linebreak\linebreak
		Define $a_h(v,w)=\sum\limits_{T \in \mathcal{T}_h} \int_T \triangledown v \cdot \triangledown w $$dx$
		\linebreak\linebreak
		And $||v||_h=\sqrt{a_h(v,v)} $
		\linebreak\linebreak
		Then, $a_h(.,.) \equiv a(.,.)$ on $V$.
		\linebreak\linebreak
		Also $a_h$ is positive-definite on $V_h$.
	}
	
	\frame{
		\frametitle{$P_1$ Nonconforming Element}
		\textbf{Theorem} If $\Omega$ is a convex polygonal domain , $f \in L^2(\Omega)$ then
		$$ ||u-u_h||_h<Ch^2|u|_{H^2(\Omega)} $$.
	}
\end{document}
